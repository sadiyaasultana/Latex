\documentclass{article}
\usepackage{ragged2e}
\usepackage{fancyhdr} 
\fancyhead{} 
\fancyhead[L]{LATEX} 
\fancyhead[R]{sadiya sultana\quad \thepage} 
\fancyhead[C]{PROGRAM 2}
\fancyfoot{} 
\fancypagestyle{plain}
\fancyfoot{} 
\fancyfoot[L]{Dept of ISE,CIT GUBBI} 
\fancyfoot[R]{2023-2024 \quad \thepage} 
\pagestyle{fancy} 
\title{\textbf{Abstract}} 
\vspace{-8ex}
\date{\today}
\begin{document} 
\maketitle
\thispagestyle{empty} 
\justify{Whenever multi-sense or non-domain specificity arises in a query it is difficult to deliver exact or approximate results to users for that query in considerable time limit. Modern search engines fetch enough similar results for a query over a data tree or a corpus by applying query approximation algorithms. The proposed approximate query answering model called Query Answering with Pointed Graphs (QAPG) achieves query approximation by evaluating the user concerned queries on proper semantic paths on an Accessible Pointed Graph (APG) relaxed with architectural clues.} \vspace{1mm} 
\justify{The proposed model formulates semantically inferred path algebra for a query and performs the path mapping with other set of path algebras of corresponding query keywords or a closely matched fuzzy set of another corresponding query keywords to find approximate queries. The concept of APG is used for weaving the paths, subsumed with the given concerned keyword set. Users are more concerned about their choice of search context so each selected attribute of the query is weighted according to the nature of data items either numerical or categorical in type.}
\vspace{1mm} 
\justify{The content similarity function is used to associate the categorical values to weighted attributes to evaluate overall content similarity. The overall similarity of the obtained paths is calculated from the association of content similarity and twig level similarity. The approximation function elegantly combines structure with contents to answer approximate queries. User preference on top-k answers are adjusted by an adjustment coefficient. The approximation function can find out a range of most relevant answers from a large number of XML data sources by tuning the adjustment coefficient.}
\justify{\textbf{Keywords:} Path algebra, Twig, APG, Architectural clues, Content similarity.} 
\end{document} 