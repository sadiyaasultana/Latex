\documentclass{article} 
\usepackage[english]{babel} 
\usepackage{ragged2e} 
\usepackage{fancyhdr} 
\fancyhead{} 
\fancyhead[L]{left header} 
\fancyhead[R]{right header \quad \thepage} 
\fancyfoot{} 
\fancyhead[l]{LATEX}
\fancyhead[r]{sadiya sultana}
\fancyhead[C]{PROGRAM 8}
\fancyfoot{} 
\fancyfoot[L]{Dept of ISE,CIT GUBBI} 
\fancyfoot[R]{2023-2024 \quad \thepage} 
\pagestyle{fancy} 
\newtheorem{theorem}{Theorem}
\newtheorem{corollary}{Corollary}[theorem] 
\newtheorem{lemma}[theorem]{Lemma} 
\newtheorem{definition}{Definition}[section]   
\begin{document} 
\section{Theorem} 
\begin{theorem}[Accessible pointed graph]
\justify{Consider an XML database D and a twig query q with only ancestor, descendant relationships in branching edges. The worst case I/O complexity is decided by the number of holistic nodes in the path algebra. The above theorem strongly supports the existence of accessible pointed graphs in a tree.} 
\end{theorem}  
\section{Corollary}
\begin{corollary} 
Corresponding corollary 
\end{corollary} 
\section{Lemma} 
\begin{lemma} 
Corresponding Lemma
\end{lemma} 
\section{Definition}
\begin{definition} 
Corresponding definition 
\end{definition} 
\end{document}